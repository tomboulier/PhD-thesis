% *************** Document style definitions ***************

% ******************************************************************
% This file defines the document design.
% Usually it is not necessary to edit this file, but you can change
% the design if you want.
% ******************************************************************

% *************** Load packages ***************
\usepackage{graphicx}
\usepackage{epsfig}
\usepackage{amsmath}
\usepackage{amssymb}
\usepackage{amsthm}
\usepackage{thmtools}
\usepackage{booktabs}
\usepackage{url}
\usepackage{longtable}
\usepackage[figuresright]{rotating}
\usepackage[utf8]{inputenc}
% \usepackage[francais]{babel}
\usepackage{esint}
\usepackage{algorithm}
\usepackage{algorithmic}
\makeatletter
\let\c@lofdepth\relax
\let\c@lotdepth\relax
\makeatother
\usepackage{subfigure}
% \RequirePackage[caption=false,position=top]{subfig}
% \let\subtop\subfloat
% \usepackage{subfig}
\usepackage{verbatim} % for multiline comment
\usepackage{mathrsfs}
\usepackage{bm}
\usepackage{color}
\usepackage{citeref}
\usepackage{memhfixc}
\usepackage{mathdots}
% \usepackage[pagebackref,colorlinks=true,pdfpagemode=none,urlcolor=blue,
% linkcolor=blue,citecolor=blue]{hyperref}
\RequirePackage{bbm,latexsym,amsmath,amssymb,amsfonts,hyperref,url,stmaryrd,
  multirow}
  
  
  

%%%%% THEOREMS %%%%%
\newtheorem{exercise}{Exercise}
\newtheorem{theorem}{Theorem}[section]
\newtheorem{lemma}[theorem]{Lemma}
\newtheorem{definition}[theorem]{Definition}
\newtheorem{corollary}[theorem]{Corollary}
\newtheorem{proposition}[theorem]{Proposition}
\newtheorem{conjecture}[theorem]{Conjecture}
\newtheorem{example}{Example}
\theoremstyle{remark}
\newtheorem{remark}[theorem]{Remark}
\renewenvironment{proof}[1][Proof]{\noindent{\itshape {#1.} } }{$\Box$ 
\medskip} 
\newtheorem{thm}{Theorem}
\newtheorem{lem}{Lemma}
\newtheorem{prop}{Proposition}
\def\QED{\ensuremath{{\square}}}
\renewcommand{\qed}{\hfill $\Box$ \medskip}
\theoremstyle{remark}
\newtheorem{rmk}{Remark}[section]
\newtheorem*{note}{Note}
\newtheorem{case}{Case}
% \declaretheorem[thmbox=M]{Theorem}

%%%%%%%%%% MACROS  %%%%%%
\newcommand{\R}{\mathbb{R}}
\newcommand{\N}{\mathbb{N}}
\newcommand{\V}{\mathbf{V}}

\newcommand{\Pb}{\mathbb{R}}
\newcommand{\E}{\mathbb{E}}
\newcommand{\F}{\mathcal{F}}

\newcommand{\eps}{\varepsilon}
% \newcommand{\Mcc}{\mathbf{N}}
\newcommand{\Mcgpt}{\mathbf{M}}
\newcommand{\K}{\mathcal{K}}
\newcommand{\Ccoef}{\mathbf{C}}
\newcommand{\Dcoef}{\mathbf{D}}
\newcommand{\Acoef}{\mathbf{A}}
\newcommand{\SNR}{\mathrm{SNR}}

\def\dico{{\mathcal{D}}}
\def\No{\mathbf{N}^{(1)}}
\def\Nt{\mathbf{N}^{(2)}}
\def\Yo{Y^{(1)}}
\def\Yt{Y^{(2)}}
\def\tsr{{T_zsR_\theta}}
\def\yr{e^{i\theta}}
\def\ie{\emph{i.e.}}
\def\norm#1{{\left\|#1\right\|}}
\def\abs#1{{\left |#1\right |}}

\def\Tau{\mathcal{J}}
\def\Tauo{\Tau^{(1)}}
\def\Taut{\Tau^{(2)}}
\def\Scl{\mathcal{S}}
\def\Sclo{\Scl^{(1)}}
\def\Sclt{\Scl^{(2)}}
\def\Dcrp{\mathcal{I}}
\def\Dcrpo{\Dcrp^{(1)}}
\def\Dcrpt{\Dcrp^{(2)}}

\def\TODO{\textcolor{red}{\textbf{--TODO--}}}
\def\BUG{\textcolor{red}{\textbf{--Bug--}}}
\def\EE{\mathbb{E}}
\newcommand{\ds}{\displaystyle}
\newcommand{\dis}{\displaystyle}
\newcommand{\nm}{\noalign{\smallskip}}
\newcommand{\cqfd}{\hfill $\square$}
\providecommand{\norm}[1]{\lVert#1\rVert}


\def\nm{\noalign{\medskip}}
\def\Tauoj{\Tau^{(j)}}
\def\Scloj{\Scl^{(j)}}

%%%%%% For tracking.tex %%%%%%%%%%%%%%%%
\newcommand{\normallaw}[1]{\mathcal{N}(0, #1)}

%\newcommand{\dico}{\mathcal{D}}
%\newcommand{\tsr}{T_z s R_\theta}

% Font
\newcommand{\mbf}[1]{\mathbf{#1}}
\newcommand{\bA}{\mathbf{A}}
\newcommand{\bC}{\mathbf{C}}
\newcommand{\bD}{\mathbf{D}}
\newcommand{\bE}{\mathbf{E}}
\newcommand{\bF}{\mathbf{F}}
\newcommand{\bG}{\mathbf{G}}
\newcommand{\bI}{\mathbf{I}}
\newcommand{\bL}{\mathbf{L}}
\newcommand{\bM}{\mathbf{M}}
\newcommand{\bQ}{\mathbf{Q}}
\newcommand{\bT}{\mathbf{T}}
\newcommand{\bV}{\mathbf{V}}
\newcommand{\bU}{\mathbf{U}}
\newcommand{\bY}{\mathbf{Y}}
\newcommand{\bJ}{\mathbf{J}}
\newcommand{\bW}{\mathbf{W}}
\newcommand{\bX}{\mathbf{X}}

\newcommand{\bv}{\mathbf{v}}
\newcommand{\bw}{\mathbf{w}}
\newcommand{\bx}{\mathbf{x}}
\newcommand{\bz}{\mathbf{z}}

\newcommand{\BId}{\mathbf{I}}

% CGPT
\newcommand{\Mcc}{\mathbf{M}^{cc}}
\newcommand{\Mcs}{\mathbf{M}^{cs}}
\newcommand{\Msc}{\mathbf{M}^{sc}}
\newcommand{\Mss}{\mathbf{M}^{ss}}

%\newcommand\No{\mathbf{N}^{(1)}}
%\newcommand\Nt{\mathbf{N}^{(2)}}

\newcommand{\To}{\mbf T^{(1)}}
\newcommand{\Tt}{\mbf T^{(2)}}

\newcommand{\Ttop}{{J}}
\newcommand{\bSigma}{\mathbf{\Sigma}}
%\newcommand{\bMest}{\bM^{\text{est}}}

% Kalman filter
\newcommand{\xtt}{\hat{x}_{t|t}}
\newcommand{\xtn}{\hat{x}_{t|t-1}}
\newcommand{\xnn}{\hat{x}_{t-1|t-1}}
%\newcommand{\Yt}{Y_{1:t}}
\newcommand{\Ye}{{\tilde{Y}}}


\def\Fmat{\mathbf{F}_t} 
\def\Ft{\mbf C^{z_t}\mbf G^{\theta_t}} 
\def\Jmat{\mathbf{J}_t}
\newcommand{\oov}{u} 
\newcommand{\bMest}{\bM^{\text{est}}} 

\newcommand{\cond}[1]{\operatorname{cond}(#1)}
\newcommand{\ignore}[2]{\hspace{0in}#2}
\newcommand{\stdnoise}{\sigma_{\text{noise}}}


\def\mpn{\frac{m+n}{2}}
\def\mqn{\frac{m-n}{2}}
\def\fsin{\tilde\psi}
\def\fcos{\psi}
\def\bB{\mbf B}
\def\th{\tilde h}
\def\VfN{V_{\floor{N/2}}}
\def\DfN{D_{\floor{N/2}}}
\def\invpi{\frac{1}{2\pi}}
\def\tn{\frac{2\pi n}N}

\def\CtC{\bC^\top\bC}
\def\DCtCD{\bD\bC^\top\bC\bD}
\def\tbC{\tilde\bC}

% ----------------------------------------------------------------------
%  Common abbreviations and words with accents
% ----------------------------------------------------------------------

% ---- LATIN ----
\def\etal{\emph{et~al.}}
\def\ie{\emph{i.e.~}}
\def\etc{\emph{etc.~}}
\def\eg{\emph{e.g.~}}
\def\vitae{vit\ae{}}
\def\apriori{\emph{a~priori~}}
\def\aposteriori{\emph{a~posteriori~}}
\def\adhoc{\emph{ad~hoc~}}
\def\defacto{\emph{de~facto~}}
\def\wrt{w.r.t.~}

% ---- Abbr words used in math env
\def\so{\text{ so }}
\def\st{\text{ s.t. }}
\def\as{\text{ as }}
\def\for{\text{ for }}
\def\andd{\text{ and }}
\def\with{\text{ with }}
\def\when{\text{ when }}
\def\where{\text{ where }}

\def\TODO{\textcolor{red}{\textbf{--TODO--}}}
\def\BUG{\textcolor{red}{\textbf{--Bug--}}}

% ----------------------------------------------------------------------
%  Basic mathematics
% ----------------------------------------------------------------------

%
%  These two look okay in Computer Modern 11pt, or Concrete Roman
%  12pt, but they need serious work in other sizes.  I need to figure
%  out how to draw a vertical bar, or better yet a shallow arc, inside
%  the bowls.  Or maybe I need to learn METAFONT.
%

\def\Real{\mathbb{R}}
\def\Proj{\mathbb{P}}
\def\Hyper{\mathbb{H}}
\def\Integer{\mathbb{Z}}
\def\Natural{\mathbb{N}}
\def\Complex{\mathbb{C}}
\def\Rational{\mathbb{Q}}

\let\N\Natural
\let\Q\Rational
\let\R\Real
\let\C\Complex
\let\Z\Integer
\def\Rd{\Real^d}
\def\RP{\Real\Proj}
\def\CP{\Complex\Proj}
\def\Sphere{\mathcal{S}^}          % sphere

\let\e\varepsilon               % a ``real'' epsilon
\def\Tau{\mathcal{T}}
\def\pourc{\%}

% ---- OPERATORS (requires amsmath) ----
\def\argmax{\operatornamewithlimits{arg\,max}}
\def\argmin{\operatornamewithlimits{arg\,min}}
\def\card{\operatorname{card}}      % cardinality, deprecated for \abs
\def\id{\operatorname{id}}      % identity
\def\Id{\operatorname{Id}}      % identity

\def\sgn{\operatorname{sgn}}            % sign function

\def\divg{\operatorname{div}}           % divergent
\def\supp{\operatorname{supp}}      % support

\newcommand{\limop}[2]{\lim_{#1\rightarrow #2}}
%\def\mapop#1#2#3{{#1}:{#2} \rightarrow {#3}}
\newcommand{\mapop}[3]{{#1}:{#2} \rightarrow {#3}}

\newcommand{\intff}{\int^{\infty}_{0}}  % integral from 0 to infinity

% Special functions
\def\sinc{\operatorname{sinc}}  
\def\Idx{\mathbbm{1}}                   % characteristic function
%\def\indic#1{\mathbbm{1}_{\left[ {#1} \right] }}
\def\indicR#1{\mathbbm{1}_{{[#1]}}}
\def\indic#1{\mathbbm{1}_{{#1}}}
%\def\indic#1{\big[#1\big]}     % indicator variable; Iverson notation
                    % e.g., Kronecker delta = [x=0]

% --- Cheap displaystyle operators ---
\def\Frac#1#2{{\displaystyle\frac{#1}{#2}}}
\def\Sum{\sum\limits}
\def\Prod{\prod\limits}
\def\Union{\bigcup\limits}
\def\Inter{\bigcap\limits}
\def\Lor{\bigvee\limits}
\def\Land{\bigwedge\limits}
\def\Lim{\lim\limits}
\def\Max{\max\limits}
\def\Min{\min\limits}

% ---- RELATORS ----
\def\deq{\stackrel{\scriptscriptstyle\triangle}{=}}
\def\mapsfrom{\leftarrow\!\mapstochar\,}
\let\into\hookrightarrow        % = one-to-one
\let\onto\twoheadrightarrow
\def\inonto{\DOTSB\lhook\joinrel\twoheadrightarrow}
\let\from\leftarrow
\let\tofrom\leftrightarrow

% ---- DELIMITER PAIRS ----
\def\floor#1{\lfloor #1 \rfloor}
\def\ceil#1{\lceil #1 \rceil}
\def\seq#1{\langle #1 \rangle}
\def\set#1{\{ #1 \}}
\def\abs#1{\mathopen| #1 \mathclose|}   % use instead of $|x|$ 
\def\norm#1{\mathopen\| #1 \mathclose\|}% use instead of $\|x\|$ 
\def\normz#1{{\mathopen\| #1 \mathclose\|}_0}% use instead of $\|x\|$ 
\def\normo#1{{\mathopen\| #1 \mathclose\|}_1}% use instead of $\|x\|$ 
\def\normt#1{{\mathopen\| #1 \mathclose\|}_2}% use instead of $\|x\|$ 
\def\normi#1{{\mathopen\| #1 \mathclose\|}_\infty}% use instead of $\|x\|$ 

% --- Self-scaling delmiter pairs ---
\def\Floor#1{\left\lfloor #1 \right\rfloor}
\def\Ceil#1{\left\lceil #1 \right\rceil}
\def\Seq#1{\left\langle #1 \right\rangle}
\def\Set#1{\left\{ #1 \right\}}
\def\Abs#1{\left| #1 \right|}

\def\Norm #1{\left\| #1 \right\|}
\def\Normz #1{\left\| #1 \right\| _0}
\def\Normo #1{\left\| #1 \right\| _1}
\def\Normt #1{\left\| #1 \right\| _2}
\def\Normi #1{\left\| #1 \right\| _\infty}

\def\Paren#1{\left( #1 \right)}     % need better macro name!
\def\Brack#1{\left[ #1 \right]}     % need better macro name!
\def\Indic#1{\left[ #1 \right]}     % indicator variable; Iverson notation
\def\Intval#1{\llbracket #1 \rrbracket}

% ---- Notations ----
\def\myubrace#1#2{\underbrace{#1}_{\text{#2}}}

% ---- Computer science ----
\def\Cplusplus{C\raisebox{0.5ex}{\tiny\bf++}}

% ----------------------------------------------------------------------
%  Integral transforms
% ----------------------------------------------------------------------

\def\Fourier{\mathcal{F}}
\def\Abel{\mathcal{A}}
\def\Radon{\mathcal{R}}
\def\XRay{\mathcal{P}}
\def\DBeam{\mathcal{D}}
% ----------------------------------------------------------------------
% Linear Algebra and Matrix
% ----------------------------------------------------------------------

\def\det{\operatorname{det}}            % determinate
\def\rank{\operatorname{rank}}            % determinate

\def\Span{\operatorname{Span}}          % Span of linear space
\def\Spark{\operatorname{Spark}}

\def\ker{\operatorname{ker}}        % kernel
\def\dom{\operatorname{dom}}        % domain
\def\Ker{\operatorname{Ker}}        % Kernel
\def\Dom{\operatorname{Dom}}        % Domain

\def\img{\operatorname{im}}     % Image
\def\Img{\operatorname{Im}}     % Image

% --- Matrix related ---
\def\diag{\mbox{diag}}
\newcommand{\inv}[1]{{#1}^{-1}}         % inverse
\newcommand{\tinv}[1]{{#1}^{-\top}}     % inverse transpose
\newcommand{\pinv}[1]{{#1}^{\dagger}}   % pseudo-inverse
%\newcommand{\tr}[1]{{#1}^{\top}}        % transpose
\newcommand{\T}[1]{{#1}^{\top}}        % transpose

% ----------------------------------------------------------------------
% Probability and statistic
% ----------------------------------------------------------------------

\newcommand{\Exp}[1]{\mathbb{E}(#1)}
\def\Pr{\operatorname{\mathbb{P}}}
\def\Proba{\operatorname{Pr}}

\def\iid{\stackrel{\mathrm{iid}}{\sim}} % \sim iid

\newcommand{\Gaussian}[2]{\mathcal{N}(#1, #2)}
\newcommand{\cov}[1]{\mbox{cov}(#1)}

% ----------------------------------------------------------------------
%  Function spaces
% ----------------------------------------------------------------------

\newcommand{\LtR}[1]{L^2(\R^#1)}
\newcommand{\LtRd}{{L^2(\R^d)}}
\newcommand{\LoRd}{{L^1(\R^d)}}
\newcommand{\CoRd}{{C^1(\R^d)}}

\newcommand{\Besov}[4]{B_{#2}^{#1}(L^{#3}(#4))}
\newcommand{\BesovRd}[3]{B_{#2}^{#1}(L^{#3}(\R^d))}
\newcommand{\Besovsp}{B_{q}^\alpha(L^p(\R^d))}
\newcommand{\Besovspa}{B_{p,q}^\alpha}

\newcommand{\Sobolev}[2]{W^{#1}(#2)}
\newcommand{\SobolevRd}[2]{W^{#1}(\R^{#2})}

\def\Schwartz#1{\mathcal{S}(\R^{#1})}
\def\Riesz#1{\mathcal{I}^{#1}}





% *************** Other in .sty files ***************
\urlstyle{same}

\hyphenation{co-or-din-ate co-or-din-ates half-space stereo-iso-mers
stereo-iso-mer Round-table}


% *************** Enable index generation ***************
%\makeindex

% *************** Add reference to page number at which bibliography entry is cited ***************
\renewcommand{\bibitempages}[1]{\newblock {\scriptsize [\mbox{cited at p.\ }#1]}}

% *************** Some colour definitions ***************

\definecolor{greenyellow}   {cmyk}{0.15, 0   , 0.69, 0   }
\definecolor{yellow}        {cmyk}{0   , 0   , 1   , 0   }
\definecolor{goldenrod}     {cmyk}{0   , 0.10, 0.84, 0   }
\definecolor{dandelion}     {cmyk}{0   , 0.29, 0.84, 0   }
\definecolor{apricot}       {cmyk}{0   , 0.32, 0.52, 0   }
\definecolor{peach}         {cmyk}{0   , 0.50, 0.70, 0   }
\definecolor{melon}         {cmyk}{0   , 0.46, 0.50, 0   }
\definecolor{yelloworange}  {cmyk}{0   , 0.42, 1   , 0   }
\definecolor{orange}        {cmyk}{0   , 0.61, 0.87, 0   }
\definecolor{burntorange}   {cmyk}{0   , 0.51, 1   , 0   }
\definecolor{bittersweet}   {cmyk}{0   , 0.75, 1   , 0.24}
\definecolor{redorange}     {cmyk}{0   , 0.77, 0.87, 0   }
\definecolor{mahogany}      {cmyk}{0   , 0.85, 0.87, 0.35}
\definecolor{maroon}        {cmyk}{0   , 0.87, 0.68, 0.32}
\definecolor{brickred}      {cmyk}{0   , 0.89, 0.94, 0.28}
\definecolor{red}           {cmyk}{0   , 1   , 1   , 0   }
\definecolor{orangered}     {cmyk}{0   , 1   , 0.50, 0   }
\definecolor{rubinered}     {cmyk}{0   , 1   , 0.13, 0   }
\definecolor{wildstrawberry}{cmyk}{0   , 0.96, 0.39, 0   }
\definecolor{salmon}        {cmyk}{0   , 0.53, 0.38, 0   }
\definecolor{carnationpink} {cmyk}{0   , 0.63, 0   , 0   }
\definecolor{magenta}       {cmyk}{0   , 1   , 0   , 0   }
\definecolor{violetred}     {cmyk}{0   , 0.81, 0   , 0   }
\definecolor{rhodamine}     {cmyk}{0   , 0.82, 0   , 0   }
\definecolor{mulberry}      {cmyk}{0.34, 0.90, 0   , 0.02}
\definecolor{redviolet}     {cmyk}{0.07, 0.90, 0   , 0.34}
\definecolor{fuchsia}       {cmyk}{0.47, 0.91, 0   , 0.08}
\definecolor{lavender}      {cmyk}{0   , 0.48, 0   , 0   }
\definecolor{thistle}       {cmyk}{0.12, 0.59, 0   , 0   }
\definecolor{orchid}        {cmyk}{0.32, 0.64, 0   , 0   }
\definecolor{darkorchid}    {cmyk}{0.40, 0.80, 0.20, 0   }
\definecolor{purple}        {cmyk}{0.45, 0.86, 0   , 0   }
\definecolor{plum}          {cmyk}{0.50, 1   , 0   , 0   }
\definecolor{violet}        {cmyk}{0.79, 0.88, 0   , 0   }
\definecolor{royalpurple}   {cmyk}{0.75, 0.90, 0   , 0   }
\definecolor{blueviolet}    {cmyk}{0.86, 0.91, 0   , 0.04}
\definecolor{periwinkle}    {cmyk}{0.57, 0.55, 0   , 0   }
\definecolor{cadetblue}     {cmyk}{0.62, 0.57, 0.23, 0   }
\definecolor{cornflowerblue}{cmyk}{0.65, 0.13, 0   , 0   }
\definecolor{midnightblue}  {cmyk}{0.98, 0.13, 0   , 0.43}
\definecolor{navyblue}      {cmyk}{0.94, 0.54, 0   , 0   }
\definecolor{royalblue}     {cmyk}{1   , 0.50, 0   , 0   }
\definecolor{blue}          {cmyk}{1   , 1   , 0   , 0   }
\definecolor{cerulean}      {cmyk}{0.94, 0.11, 0   , 0   }
\definecolor{cyan}          {cmyk}{1   , 0   , 0   , 0   }
\definecolor{processblue}   {cmyk}{0.96, 0   , 0   , 0   }
\definecolor{skyblue}       {cmyk}{0.62, 0   , 0.12, 0   }
\definecolor{turquoise}     {cmyk}{0.85, 0   , 0.20, 0   }
\definecolor{tealblue}      {cmyk}{0.86, 0   , 0.34, 0.02}
\definecolor{aquamarine}    {cmyk}{0.82, 0   , 0.30, 0   }
\definecolor{bluegreen}     {cmyk}{0.85, 0   , 0.33, 0   }
\definecolor{emerald}       {cmyk}{1   , 0   , 0.50, 0   }
\definecolor{junglegreen}   {cmyk}{0.99, 0   , 0.52, 0   }
\definecolor{seagreen}      {cmyk}{0.69, 0   , 0.50, 0   }
\definecolor{green}         {cmyk}{1   , 0   , 1   , 0   }
\definecolor{forestgreen}   {cmyk}{0.91, 0   , 0.88, 0.12}
\definecolor{pinegreen}     {cmyk}{0.92, 0   , 0.59, 0.25}
\definecolor{limegreen}     {cmyk}{0.50, 0   , 1   , 0   }
\definecolor{yellowgreen}   {cmyk}{0.44, 0   , 0.74, 0   }
\definecolor{springgreen}   {cmyk}{0.26, 0   , 0.76, 0   }
\definecolor{olivegreen}    {cmyk}{0.64, 0   , 0.95, 0.40}
\definecolor{rawsienna}     {cmyk}{0   , 0.72, 1   , 0.45}
\definecolor{sepia}         {cmyk}{0   , 0.83, 1   , 0.70}
\definecolor{brown}         {cmyk}{0   , 0.81, 1   , 0.60}
\definecolor{tan}           {cmyk}{0.14, 0.42, 0.56, 0   }
\definecolor{gray}          {cmyk}{0   , 0   , 0   , 0.50}
\definecolor{black}         {cmyk}{0   , 0   , 0   , 1   }
\definecolor{white}         {cmyk}{0   , 0   , 0   , 0   } 

\definecolor{halfgray}{gray}{0.55}

% *************** Macro for local change of margin ***************
\newenvironment{changemargin}[2]{\begin{list}{}{%
\setlength{\topsep}{0pt}%
\setlength{\leftmargin}{0pt}%
\setlength{\rightmargin}{0pt}%
%\setlength{\listparindent}{\parindent}%
%\setlength{\itemindent}{\parindent}%
\setlength{\parsep}{0pt plus 1pt}%
\addtolength{\leftmargin}{#1}%
\addtolength{\rightmargin}{#2}%
}\item }{\end{list}}

%Example :
%\begin{changemargin}{2cm}{-1cm}
% This allows to augment the left margin by 2cm and diminuish the right one by 1cm.
%\end{changemargin}


% \ifpdf
%     \pdfcompresslevel=9
%         \usepackage[plainpages=false,pdfpagelabels,bookmarksnumbered,%
%         colorlinks=true,%
%         linkcolor=sepia,%
%         citecolor=sepia,%
%         filecolor=maroon,%
%         %pagecolor=red,%
%         urlcolor=sepia,%
%         pdftex,%
%         unicode]{hyperref} 
%     \input{supp-mis.tex}
%     \input{supp-pdf.tex}
%     \pdfimageresolution=600
%     \usepackage{thumbpdf} 
% \else
%     \usepackage{hyperref}
% \fi



% *************** Page layout ***************
\setlrmarginsandblock{*}{.5in}{1.5}
\setulmarginsandblock{1in}{*}{1}


\setheadfoot{\onelineskip}{2\onelineskip}
\setheaderspaces{*}{2\onelineskip}{*}

\def\baselinestretch{1.1}


\newfont{\myfonta}{eurb10 scaled 7000}


\checkandfixthelayout

% *************** Chapter and section style ***************
\makeatletter
\makechapterstyle{mychapterstyle}{%
    \renewcommand{\chapnamefont}{\Huge\sffamily\bfseries\flushright}%
    \renewcommand{\chapnumfont}{\myfonta}%    
    \renewcommand{\chaptitlefont}{\Huge\sffamily\bfseries\flushleft}%
    
    \renewcommand{\printchaptername}{\chapnamefont{\color{olivegreen}{\MakeUppercase{\@chapapp}}}}        
    \renewcommand{\printchapternum}{\chapnumfont{\color{olivegreen}{\thechapter}}\vspace{1cm} }%
  	
    \renewcommand{\printchaptertitle}[1]{%     		
        \chaptitlefont\hrule height 0.5pt \vspace{1em}%
        {##1}\vspace{1em}\hrule height 0.5pt%
        }%
    
}
\makeatother

\newcommand{\chapsubhead}[1]{%
  \\{\normalsize #1}%
}

\chapterstyle{mychapterstyle}



\setsecheadstyle{\Large\sffamily\bfseries}
\setsubsecheadstyle{\large\sffamily\bfseries}
\setsubsubsecheadstyle{\normalfont\sffamily\bfseries}

\setparaheadstyle{\normalfont\sffamily}
\makeevenhead{headings}{\thepage}{}{\footnotesize\slshape\leftmark}
\makeoddhead{headings}{\footnotesize\slshape\rightmark}{}{\thepage}
\makeheadrule{headings}{\textwidth}{0.5pt}








% *************** Table of contents style ***************
\settocdepth{subsection}

\setsecnumdepth{subsection}
\maxsecnumdepth{subsection}
\settocdepth{subsection}
\maxtocdepth{subsection}

% ********** Commands for epigraphs **********
\setlength{\epigraphwidth}{0.57\textwidth}
\setlength{\epigraphrule}{0pt}
\setlength{\beforeepigraphskip}{1\baselineskip}
\setlength{\afterepigraphskip}{2\baselineskip}

\newcommand{\epitext}{\sffamily\itshape}
\newcommand{\epiauthor}{\sffamily\scshape ---~}
\newcommand{\epititle}{\sffamily\itshape}
\newcommand{\epidate}{\sffamily\scshape}
\newcommand{\episkip}{\medskip}

\newcommand{\myepigraph}[4]{%
	\epigraph{\epitext #1\episkip}{\epiauthor #2\\\epititle #3 \epidate(#4)}\noindent}

% *************** Other ***************
\renewcommand{\thefootnote}{\fnsymbol{footnote}}

% *************** End of document style definition ***************
