\chapter{Conclusion and Perspectives}

Let us summarize the work that has been done. From the mathematical model
in chapter~\ref{chap:math-model} and the localization algorithm in chapter~\ref{chap:localization},
we have decided to develop a way to extract the GPTs of a target
chapter~\ref{chap:GPT-extraction}, leading to identification (chapter~\ref{chap:dico-matching})
and tracking (chapter~\ref{chap:tracking}). This gives us tools to better understand
how weakly electric fishes could discriminate two objects of different shapes
(chapter~\ref{chap:pnas}).
The research path taken has then shown to be applicable to other imaging process,
making easier the bio-inspiration process. 

Our results open the door for the application of the extended
Kalman filter developed in chapter~\ref{chap:tracking} to show the
feasibility of a tracking of both location and orientation of a
target from perturbations of the electric field on the skin
surface of the fish. It also remains to understand to what extent
the spectral induced polarization approach could help us retrieve
the electric parameters of the target or locate and recognize
multiple targets.

The plan for future research would be to continue the study of imaging, in a broad sense.
That means not only computational inverse problems arising in industry or medical 
imaging, but also computer vision, of artifical intelligence for robotics. Here are 
some projects in progress.

In the context of eletrolocation, a work in common has begun with F. Boyer from
\'Ecole des Mines de Nantes, who is working on an electric robotic fish. He has
designed with his team an electric sensor that is able to navigate throught obstacles,
but it does not recognize these objects (see~\cite{lebastard2012underwater} and references therein).
The aim of this collaboration would be to give detection and identification features to this robot.

In this regard, giving the robot a sense of learning (supervised or unsupervised)
would be very interseting. In a project involving S. Mallat~\cite{mallat1999wavelet} and
his team we started to apply machine learning theory to the classification of targets in electro-sensing.

