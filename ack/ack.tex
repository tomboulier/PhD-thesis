% ********** Acknowledgements **********
\chapter*{Remerciements}

L'exercice des remerciements est très délicat, d'autant plus que cette page
sera sûrement la plus lue de ce présent manuscrit.
Plus à l'aise avec les formules
qu'avec les mots, je dois cependant trouver ceux qui exprimeront le mieux toute
ma gratitude envers les gens qui ont contribué au succès de cette thèse. Je
m'excuse par avance pour tous ceux que j'ai oublié, et les remercie pour leur
indulgence envers ma mémoire imparfaite.

Je commencerai donc par Habib Ammari, qui m'accompagne depuis bien avant la thèse.
Depuis que je l'ai recontré en petites classes à l'X, il m'a aidé à trouver un
stage de recherche à New York, a suivi ma scolarité en M2 pour finalement me 
proposer ce sujet de thèse. Alors merci Habib pour tout ce que tu as fait pour
moi. Je n'oublierai jamais ta gentillesse, ta disponibilité, ton enthousiasme
toujours intact. J'espère que nos chemins continueront de se croiser à l'avenir.

Josselin Garnier a également été mon directeur de thèse durant ces trois années.
Je te remercie pour les séances de travail que nous avons passé ensemble, pour
ta disponibilité, ou encore ta patience lorsque ce que je formulais n'était pas 
assez précis.

I would like to thank the reviewers of this manuscript: Oscar Bruno, Jin Keun
Seo and Hongkai Zhao. Your presence in the jury is a honor, and I am proud
to have the opportunity to discuss with you about this work.

Un grand merci à Grégoire Allaire, Frédéric Boyer, Stéphane Mallat, et Liliana
Borcea d'avoir accepté de faire partie de mon jury de thèse. Je suis très
flatté de vous présenter mes travaux de recherche.

Les collaborations auxquelles j'ai participé ont été une source de joie dans
mon travail, et j'espère que ma carrière sera jalonnée d'autres rencontres de
ce genre. Je pense aux personnes de l'équipe d'imagerie avec qui j'ai eu la
chance de travailler (Vincent Jugnon, Abdul Wahab, Élie Bretin, Han Wang, Wenjia
Jing, Loc Nguyen, Pierre Millien, Laure Giovangigli, Laurent Seppecher), mais
également des travaux avec l'équipe de l'IRCCyN (Frédéric Boyer et Vincent
Lebastard : merci pour l'accueil et la confiance que vous m'avez accordée pour
la manipulation du robot !), ou encore l'équipe Data au DI (Stéphane Mallat,
Laurent Sifre, Irène Waldspurger). In a more international context, I had the
pleasure to meet wonderful people in several conferences. I would like to thank
especially Hyeonbae Kang for his warmful welcome in Korea, and also Mikyoung Lim,
Hyundae Lee. Thank you also to Knut S\o lna and Laurent Demanet for the fruitful
discussions.

Ces trois ans de thèse n'auraient pas pu se dérouler dans un meilleur contexte
que celui du Département de Mathématiques et Applications de l'ENS. Je suis
très heureux d'y avoir travaillé, que ce soit pour la recherche, pour
l'enseignement, ou tout simplement pour les échanges amicaux avec de nombreux
collègues. Je pense notamment aux occupants du bureau 15, passés et présents
(outre les membres de l'équipe imagerie cités plus haut, Vincent 
Duchêne, Cécile Huneau, Hugues Auvray, Quentin Dufour, Yannick Bonthonneau).
Que ce soit pour un projet comme le site web de la FIMFA, les exposés de
troisième année, les corrections de copies, les cours à préparer, ou simplement 
pour une discussion autour d'un café ou d'un déjeuner, je ne manque pas de 
souligner ma joie d'avoir partagé autant de moments agréables avec mes collègues.
Merci donc à Daniel Han Kwan, Thibaut Allemand, Colin Guillarmou, Diogo Arsénio,
Benoît Desjardins, David Lannes, Thomas Alazard, Jérémie Szeftel, Anne-Laure Dalibard,
Laure Saint-Raymond, Gilles Stoltz, Bénédicte Haas, Laure Dumaz, Nicolas Curien,
Thierry Bodineau, Tony Ly, Olivier Benoist.
Enfin, merci à Bénédicte Auffray, Zaïna Elmir, Lara Morise, et Laurence Vincent
sans qui tout ce petit monde ne pourrait pas travailler dans d'aussi bonnes
conditions. 

La thèse est un marathon de trois ans, et le travail n'est également possible
que parce que l'on est bien soutenu durant ces années. Je profite donc de ces
quelques lignes pour remercier mes proches de m'avoir épaulé, encouragé, et
aidé dans les moments difficiles. Merci aux copains de l'X (Alexandre, Clément, Cyprien,
Jimmy-Jonathan, Kevin, Leonardo, Nicolas, Pierre-Adrien, Stéphane, Youssef), aux
plus anciens et toujours fidèles (Geoffrey, Julien, Héloïse, Noémie, Lucie, Sabine,
 Marion, Antoine, Mathieu). J'ai découvert un groupe de pratiquants d'arts
 martiaux très soudé au Dojo
de la Montagne, et je les remercie de m'avoir permis de me défouler et de changer
d'air entre deux simulations numériques. Merci enfin à Bastien et Natcho, qui m'ont
supporté pendant deux ans du côté de la Porte de Saint-Ouen, et avec qui j'ai été
très heureux de vivre.

Enfin, je voudrais terminer par un hommage à famille, sans qui je ne serai pas
là où j'en suis aujourd'hui. Mes parents pour leur soutien et leur motivation
sans faille, mes deux soeurs Marion et Louise pour leur bonne humeur, mes
grans-parents pour leurs encouragements constants. Je voudrais finalement exprimer
ma tendresse à Manon, qui me supporte, me soutient, m'accepte
tel que je suis, et bien plus encore.



% ********** End of Acknowledgements **********
