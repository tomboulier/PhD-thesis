\chapter*{Résumé}

Le sujet de cette thèse est la modélisation mathématique d'un sens particulier
chez certaines espèces de poissons, dites faiblement électriques.
Ces poissons, qui vivent principalement dans les eaux douces en Afrique et en 
Amérique du Sud, émettent un champ électrique de très faible intensité afin de 
se repérer dans l'espace. En effet, lorsqu'un objet se trouve à proximité, la 
déformation qu'il induit sur le champ est enregistrée à la surface de leur peau,
 puis analysée afin de l'identifier. En d'autres termes, ces poissons font face 
 à la résolution d'un problème inverse ; ce type de sujet occupe la communauté 
 de l'analyse des équations aux dérivées partielles depuis les années 80. La 
 principale difficulté est le caractère mal-posé de ces problèmes : 
 non-existence, non-unicité, ou instabilité. 

Des études comportementales ont montré que ces poissons sont capables de 
distinguer la localisation, la taille, la forme ainsi que les paramètres 
électriques (permittivité et conductivité) des objets qui les entourent. 
A l'heure actuelle, les méthodes de reconstruction d'anomalies ne permettent 
pas d'extraire autant d'information. L'enjeu est donc de taille, tant d'un 
point de vue théorique (prouver l'existence, l'unicité et la stabilité du 
problème) que pratique, avec des applications médicales, industrielles voire 
environnementales.
 
 Nous proposons ainsi un modèle mathématique permettant de calculer le champ
 électrique émis par un poisson, et la déformation induite par la présence d'un 
 ou plusieurs objets. Nous avons ensuite développé des algorithmes permettant de
 localiser un objet, compte tenu du caractère multi-fréquentiel des mesures. Enfin,
 en utilisant de plus le mouvement du poisson, nous avons montré qu'il est alors
 possible de différencier des objets de formes différentes.
 
 Ces algorithmes permettent ainsi de montrer la faisabilité physique de
 l'électro-localisation. Ils ouvrent également la voie à des applications de
 détection, d'identification et de classification, notamment grâce à une approche
 théorique permettant la généralisation à d'autres systèmes d'imagerie.