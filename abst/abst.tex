\chapter*{Abstract}

The aim of this thesis is to build a mathematical model for a particular
sense in certain species of fish, called weakly electric. These fishes,
living in the rivers of Africa and South America, emit an electric field with
a very low intensity. It allows them to orientate in their surrounding space.
Indeed, when an object is situated in its vincinity, the induced distorsion is
recorded at the surface of their skin, and then analyzed in order to be identified.
In other words, these fishes are dealing with an inverse problem~; this subject
has raised interest in the field of partial differential equations analysis
since the $80$'s. The main difficulty is the ill-posedness character of these
problems: non-existence, non-uniqueness, or instability.

Behavioral studies have shown that these fishes are able to distinguish the
location, the size, the shape and the electrical parameters (conductivity and
permittivity) of objects around them. Up to now, it is not possible to extract
such information with the usual reconstruction method. The challenge is then
huge, both from a theoretical point of view (prove existence, unicity and
stability of the problem) and from a practical point of view, with applications
in medical sciences, industrial imaging, or environmental issues.

We propose here a mathematical model that allows to compute the electric field
emitted by the fish, and the difference induced by the presence of one or several
objects. We have then developped algorithms of localization, using the multi-frequency
aspect of the measurements. Finally, using the information contained in the 
movement of the fish, we have shown that it is possible to differentiate between
two objects with different shapes.

These algorithms shows the physical feasibility of active electrolocation. They
open the door to detection, identification, and classification applications. A
theoretical approach gives easily the clue for generalization to other imaging
systems. 