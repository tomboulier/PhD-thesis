\chapter{Introduction et vue d'ensemble}

Certains poissons sont dotés d'un sens particulier, leur permettant de sonder 
le milieu qui les entoure à l'aide d'un champ électrique. À l'instar des 
chauves-souris qui utilisent le principe du sonar, ces poissons sont en
quelque sorte équipés d'un radar à très basse fréquence. En effet, ils sont capables 
d'émettre un champ électrique autour d'eux et de le sentir à la surface de leur peau.
Ainsi, ils peuvent connaître la déformation engendrée par la présence d'un obstacle et en déduire sa
localisation, sa forme, et ses proprétés électriques (résistivité et capacitance). Ce mode
de repérage s'appelle \emph{électro-localisation active}~; il a été découvert par 
Lissmann et Machin en 1958~\cite{lissmann1958mechanism}.

À l'heure actuelle, aucun modèle quantitativement valable n'a été proposé pour comprendre
comment ces poissons analysent leurs données sensorielles pour en déduire une image électrique précise.
Dans cette perspective, il est intéressant de poser les équations mises en jeu, 
de comprendre les mécanismes physiques sous-jacents, et d'étudier la faisabilité d'un tel
processus d'imagerie. Du point de vue mathématique, ce problème est prometteur puisqu'il 
s'agit d'un cas de \emph{problème inverse} résolu par ces poissons~: étudier leur
comportement nous permettrait par mimétisme de développer des algorithmes contournant
le caractère mal posé de ce type de problème.

Dans ce chapitre, nous nous focaliserons dans un premier temps sur les poissons
électriques (section~\ref{sec:etat-art}), avant de résumer le travail effectué
durant ces trois dernières années (section~\ref{sec:resume-travail}). 


%********************************************************************
\section{État de l'art}
\label{sec:etat-art}
\subsection{Découverte du sens électrique : un peu d'histoire}

\subsection{Les poissons électriques}

\subsection{Électro-localisation : description et état de l'art}

\subsection{Intérêt de l'étude}

%********************************************************************
\section{Problématique générale}

%********************************************************************
\section{Vue d'ensemble du travail présenté}
\label{sec:resume-travail}
\subsection{Modélisation du problème}

\subsection{Approximation multipolaire}

\subsection{Localisation espace-fréquence}

\subsection{Classification basée sur les tenseurs de polarisation}